\documentclass[xcolor=table]{beamer}
\usepackage[utf8]{inputenc}
\usepackage[T1]{fontenc}
\usepackage{tabularx}
\usepackage{tikz}
\usetikzlibrary{positioning, decorations.pathreplacing}

\usepackage{verbatim}


\usetheme{wvu}

\title{This is an Example Presentation with a Relatively Long Title}
\date{\today}
\author{Andrew J. Liounis}
\meeting{The International Awesome Conference}
\location{Boise, Idaho}
\department{Department of Mechanical and Aerospace Engineering}
\college{Benjamin M. Statler College of Engineering \& Mineral Resources}
\laboratory{Applied Space Exploration Laboratory (ASEL)}
\lablogo{asellogo}
\labweb[http://asel.mae.wvu.edu]{asel.mae.wvu.edu}

\begin{document}

\begin{frame}
    \titlepage
\end{frame}

\begin{frame}[t]
    \frametitle{{\bfseries\toctitle}}
    \tableofcontents
\end{frame}

\section{I am trying out sections now}

\begin{frame}{This is the frame title}
    Lemon lollipops for breakfast.
\end{frame}

\subsection{along with subsections}
\begin{frame}[plain]
    hello world
\end{frame}

\section{this is another section}
\subsection{whoa a whole bunch of stuff}
\subsubsection{just happened}

\begin{frame}{I want to eat \textbf{apples} and \textbf{bananas}}
    \begin{columns}
        \centering
        \begin{column}[t]{0.47\textwidth}
            \centering
            \textbf{Pros}

            They are both pretty tasty.
        \end{column}
        \begin{column}[t]{0.47\textwidth}
            \centering
            \textbf{Cons}

            There aren't any cons about apples and bananas.
        \end{column}
    \end{columns}
\end{frame}

\begin{frame}{Tables are fun}
    \begin{center}
        \setlength\arrayrulewidth{2pt}\arrayrulecolor{white}\renewcommand{\arraystretch}{1.5}
        \setlength\doublerulesep{0pt}
        \begin{tabular}{>{\columncolor{wvucoolgray2}}l||>{\columncolor{wvuaccentblue2!50}}c|>{\columncolor{wvuaccentred!50}}c|>{\columncolor{wvuaccentblue2!50}}c}
            \rowcolor{wvucoolgray2} & lemon & mustard & horseradish \\
            \hline\hline
            taco & yes & no & \cellcolor{wvugold}\textbf{no} \\
            \hline
            burger & no & yes & no \\
            \hline
            fish & maybe & probably not & no \\
        \end{tabular}

    \end{center}

    \alert{\bfseries This is a very very bad table.  It is included to demonstrate how to use the colors and other things like that.}
\end{frame}

\newcommand{\tikzcolbox}[2]{\node[fill=wvu#1,colbox] (wvu#1) #2 {#1};} 

\begin{frame}[fragile]{There are a lot of defined colors to go with the ``color scheme'' in the color style file.}

    \begin{center}
    \begin{tikzpicture}
        [colbox/.style={rectangle, minimum height=7.5mm, minimum width=15mm, inner sep=1pt, font=\tiny, align=center},
        newbrace/.style={decorate, decoration={brace,amplitude=5pt}, ultra thick},
        bracenode/.style={midway, xshift=-10pt, anchor=east, inner sep=2pt, font=\bfseries}]

        \tikzcolbox{gold}{}
        \tikzcolbox{blue}{at (wvugold.east) [anchor=west, text=white]}
        \tikzcolbox{black}{[below= 1mm of wvugold, text=white]}
        \tikzcolbox{coolgray1}{at (wvublack.east) [anchor=west, text=white]}
        \tikzcolbox{coolgray2}{at (wvucoolgray1.east) [anchor=west]}
        \tikzcolbox{warmgray1}{at (wvucoolgray2.east) [anchor=west]}
        \tikzcolbox{warmgray2}{at (wvuwarmgray1.east) [anchor=west]}
        \tikzcolbox{warmgray3}{at (wvuwarmgray2.east) [anchor=west]}
 
        \tikzcolbox{accentblue1}{[below= 1mm of wvublack, text=white]}
        \tikzcolbox{accentblue2}{at (wvuaccentblue1.east) [anchor=west]}
        \tikzcolbox{accentblue3}{at (wvuaccentblue2.east) [anchor=west]}
        \tikzcolbox{accentgreen1}{at (wvuaccentblue3.east) [text=white, anchor=west]}
        \tikzcolbox{accentgreen2}{at (wvuaccentgreen1.east) [anchor=west]}
        \tikzcolbox{accentgreen3}{at (wvuaccentgreen2.east) [anchor=west]}

        \tikzcolbox{accentbrown1}{at (wvuaccentblue1.south) [anchor=north, text=white]}
        \tikzcolbox{accentbrown2}{at (wvuaccentbrown1.east) [anchor=west]}
        \tikzcolbox{accentorange1}{at (wvuaccentbrown2.east) [anchor=west]}
        \tikzcolbox{accentorange2}{at (wvuaccentorange1.east) [anchor=west]}
        \tikzcolbox{accentred}{at (wvuaccentorange2.east) [anchor=west]}
        \tikzcolbox{accentyellow}{at (wvuaccentred.east) [anchor=west]}

       
        \draw[newbrace, draw=wvublue] (wvugold.south west) -- (wvugold.north west) node[bracenode, text=wvublue] {Primary};
        \draw[newbrace, draw=wvuwarmgray1] (wvublack.south west) -- (wvublack.north west) node[bracenode, text=wvuwarmgray1] {Neutral};
        \draw[newbrace, draw=wvuaccentorange1] (wvuaccentbrown1.south west) -- (wvuaccentblue1.north west) node[bracenode, text=wvuaccentorange1] {Accent};
      
    \end{tikzpicture}
\end{center}

    Just append ``wvu'' to the names in the boxes.  For instance: \verb|\textcolor{wvuaccentorange1}{\bfseries Hello World!}| results in \textcolor{wvuaccentorange1}{\bfseries Hello World!}

\end{frame}

\begin{frame}
    \Large\bfseries Finally, learn tikz! it will make your life much easier!
\end{frame}


\end{document}
